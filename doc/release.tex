\documentclass{article}
\def\PAL{\mbox{C{\texttt{-{}-}}}}
% l2h substitution PAL C--
\title{Release notes for the Quick~{\PAL} compiler}
\author{The Quick~{\PAL} Team}

\begin{document}

\maketitle

\section{Notes for release 20031021}
\begin{itemize}

\item
The compiler now generates runtime data. The native runtime system now
lives in the runtime directory.

\item
The interpreter interface has been changed slightly to be more
consistent with the native runtime interface. However, there are still
a few differences between the two interfaces. Interpreter functions
take pointers to activation structures where the native runtime takes
pointers to activation structures.

\item
The default register allocator has been changed to DLS. If you need
the graph coloring allocator, you should turn off generation of runtime
data. i.e. type [[qc-- backend=Backend.x86 backend.emit_data=nil ...]]

\end{itemize}

\section{Notes for more recent releases}

\begin{itemize}
\item
If two different {\PAL} units have different global-variable
declarations, the error manifests as a failure to link.
\end{itemize}


\section{Notes for release 20030711}


This is the first major release that includes a native-code back end.
The release is believed consistent with the pre-2.0 manual (CVS
revision~1.75) with exceptions noted below.
\begin{itemize}
\item
The front end does not implement the \texttt{switch} statement.
\item
The \texttt{import} statement requires a type, although this
requirement is not documented in the manual.
\item
The native x86 back end successfully compiles the test suite
for the \texttt{lcc} C~compiler.
The back end has significant limitations:
\begin{itemize}
\item
Integer variables should be 8, 16, or 32~bits.
Integer and logical operations should be on 32-bit values only.
The back end supports the operations you would find in a C~compiler,
but we can try to add others on request.
\item
Floating-point operations may be at 32 or 64~bits.
80-bit floating-point is possible but not tested.
The back end supports the operations you would find in a C~compiler,
but we can try to add others on request.
\item 
Mixed integer and floating-point operations have not been tested
thoroughly. 
\item
The back end does not support multiprecision arithmetic or overflow
detection. 
\end{itemize}
\item
The release also includes back ends for Alpha and MIPS R3000, but these are
incomplete and have been used only to test implementations of calling
conventions. 
\item
There is as yet no run-time system to go with the native-code
compiler.
If you need a run-time system, you must use the interpreter.
\item
 Code quality is poor; as a matter of policy, we are postponing
       work on optimization in order to bring you a run-time system.
\item
The default register allocator is \emph{extremely} slow.
You may want to use the \texttt{dls} register allocator. This
allocator is dramatically faster than our implementation of the
default register allocator. You may invoke the \texttt{dls} allocator on the
command line, as in the following example:
\begin{verbatim} 
  qc-- Backend.x86.ralloc=Ralloc.dls hello.c--
\end{verbatim}
At present, the \texttt{dls} allocator has bugs that prevent it from
compiling some programs.
\end{itemize}

\end{document}
