\documentclass{article}
\def\PAL{\mbox{C{\texttt{-{}-}}}}
\title{Release notes for the Quick~{\PAL} compiler}
\author{The Quick~{\PAL} Team}

\begin{document}

\maketitle

\section{Notes for release 20030711}

This is the first major release that includes a native-code back end.
\begin{itemize}
\item
The front end does not implement the \texttt{switch} statement.
\item
The \texttt{import} statement requires a type, although this
requirement is not documented in the manual.
\item
The native x86 back end successfully compiles the test suite
for the \texttt{lcc} C~compiler.
The back end has significant limitations:
\begin{itemize}
\item
Integer variables should be 8, 16, or 32~bits.
Integer and logical operations should be on 32-bit values only.
The back end supports the operations you would find in a C~compiler,
but we can try to add others on request.
\item
Floating-point operations may be at 32 or 64~bits.
80-bit floating-point is possible but not tested.
The back end supports the operations you would find in a C~compiler,
but we can try to add others on request.
\item 
Mixed integer and floating-point operations have not been tested
thoroughly. 
\item
The back end does not support multiprecision arithmetic or overflow
detection. 
\end{itemize}
\item
There is as yet no run-time system to go with the native-code
compiler.
If you need a run-time system, you must use the interpreter.
\item
The default register allocator is \emph{extremely} slow.
You may want to use the \texttt{dls} register allocator. This
allocator is dramatically faster than our implementation of the
default register allocator. You may invoke the \texttt{dls} allocator on the
command line, as in the following example:
\begin{verbatim} 
  qc-- Backend.x86.ralloc=Ralloc.dls hello.c--
\end{verbatim}
At present, the \texttt{dls} allocator has bugs that prevent it from
compiling some programs.
\end{itemize}

\end{document}
